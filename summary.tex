\documentclass[]{article}


\title{Master Project}

\begin{document}
%%%%%%%%%%%%%%%%%
%     Title     %
%%%%%%%%%%%%%%%%%
\title {Master Project}
\author{Xiaonan Jian, Jingjing Li, Geyu Meng, Zi YE}
\date{XX-XX-2022}
\maketitle
\thispagestyle{empty}

%%%%%%%%%%%%%%%%%%%%
%     Abstract     %
%%%%%%%%%%%%%%%%%%%%
\newpage
\begin{abstract}
 This is the abstract.  
\end{abstract}
\thispagestyle{empty}
%%%%%%%%%%%%%%%%%%%
%     Content     %
%%%%%%%%%%%%%%%%%%%
\newpage
\pagenumbering{roman}
\tableofcontents

%%%%%%%%%%%%%%%%%%%%%
%     Main Body     %
%%%%%%%%%%%%%%%%%%%%%
%%%%%%%%%%%%%%%%%%%%%%%%
%     Introduction     %
%%%%%%%%%%%%%%%%%%%%%%%%
\newpage
\pagenumbering{arabic}
\section{Introduction}


%%%%%%%%%%%%%%%%%%%%%%%%
%     Related Work     %
%%%%%%%%%%%%%%%%%%%%%%%%
\newpage
\section{Related Work}

%%%%%%%%%%%%%%%%%%%%%%
%     Background     %
%%%%%%%%%%%%%%%%%%%%%%
\newpage
\section{Background}

%%%%%%%%%%%%%%%%%%%%%
%     Algorithm     %
%%%%%%%%%%%%%%%%%%%%%
\section{Algorithm}
\subsection{Character-based}
We would use Levenshtein distance, which includes three types of edit operations: inserting a character, deleting a character, and replacing one character. In general, the smaller the edit distance, the greater the similarity between the two strings.
\\\\Here are the basic steps of the algorithm. 
\begin{enumerate}
    \item Construct a matrix with rows m+1 and columns n+1, which is used to store the number of operations that need to be performed to complete a certain conversion, and the number of operations that need to be performed to convert the string s[1..n] to the string t[1..m] is the value of matrix[n][m].

    \item Initialize the first row of the matrix to be 0 to n and the first column to be 0 to m.

    Matrix[0][j] denotes the value of column j-1 of row 1. This value indicates the number of operations that need to be performed to convert the string s[1...0] to t[1...j]. It is obvious that converting an empty string to a string of length j requires only j add operations, so the value of matrix[0][j] should be j, and the other values and so on.
    
    \item Check each s[i] character from 1 to n.

    \item Check each s[i] character from 1 to m.

    \item Compare each character of string s and string t by two, if equal, let cost be 0, if not equal, let cost be 1.

    \item
    \begin{enumerate}
        \item If we can convert s[1..i-1] to t[1..j] inside k operations, then we can remove s[i] and then do these k operations, so a total of k+1 operations are needed.

        \item If we can convert s[1..i] to t[1..j-1] within k operations, that is, d[i,j-1]=k, then we can add t[j] to s[1..i], so that a total of k+1 operations are needed.

        \item If we can convert s[1..i-1] to t [1..j-1] in k steps, then we can convert s[i] to t[j] such that s[1..i] == t[1..j] is satisfied, which also requires a total of k+1 operations. 

        \item Since we want to obtain the minimum number of operations, we finally also need to compare the number of operations of these three cases, taking the minimum value as the value of d[i,j]. Then repeat the step of 3,4,5,6, and the final result is in d[n,m].
    \end{enumerate}
\end{enumerate}



\subsection{Crossparsing}



\subsection{Token-based}
Here is our process:
\begin{enumerate}
    \item We combine the sentence to a long string, and add two * symbols in the head and tail of the sentence separately.
    For example: "I have a dog" would become "**Ihaveadog**"

    \item We use the 3-grams method to split the string into many three-character pieces and set these pieces into a list.

    \item We would compare the similarity of these lists using the overlap coefficient, Jaccard coefficient, and Dice's coefficient.
    
\end{enumerate}

\noindent
The overlap coefficient:
\begin{center}
\begin{equation}
        \operatorname{sim}_{\text {overlap }}\left(s_{1}, s_{2}\right)=\frac{\left|\operatorname{tok}\left(s_{1}\right) \cap \operatorname{tok}\left(s_{2}\right)\right|}{\min \left(\left|\operatorname{tok}\left(s_{1}\right)\right|,\left|\operatorname{tok}\left(s_{2}\right)\right|\right)}
\end{equation}
\end{center}
\noindent
The Jaccard coefficient:
\begin{center}
\begin{equation}
        \operatorname{sim}_{jaccard}\left(s_{1}, s_{2}\right)=\frac{\left|\operatorname{tok}\left(s_{1}\right) \cap \operatorname{tok}\left(s_{2}\right)\right|}{\left|\operatorname{tok}\left(s_{1}\right) \cup \operatorname{tok}\left(s_{2}\right)\right|}
\end{equation}
\end{center}
\noindent
The Dice's coefficient:
\begin{center}
\begin{equation}
        \operatorname{sim}_{\text {dice }}\left(s_{1}, s_{2}\right)=\frac{2 \times\left|\operatorname{tok}\left(s_{1}\right) \cap \operatorname{tok}\left(s_{2}\right)\right|}{\left|\operatorname{tok}\left(s_{1}\right)\right|+\left|\operatorname{tok}\left(s_{2}\right)\right|}
\end{equation}
\end{center}

%%%%%%%%%%%%%%%%%%%%%%%%%%%%%%%%%%%
%     Experimental Evaluation     %
%%%%%%%%%%%%%%%%%%%%%%%%%%%%%%%%%%%
\newpage
\section{Experimental Evaluation}

%%%%%%%%%%%%%%%%%%%%%%%%%%%%%%%%%%%%%
%     Conclusion and Discussion     %
%%%%%%%%%%%%%%%%%%%%%%%%%%%%%%%%%%%%%
\newpage
\section{Conclusion and Discussion}


%%%%%%%%%%%%%%%%%%%%%%%%%%%%
%     the Bibliography     %
%%%%%%%%%%%%%%%%%%%%%%%%%%%%
\newpage
\begin{thebibliography}{99}

\bibitem{1} Helmer, S., Augsten, N., & Böhlen, M. (2012). Measuring structural similarity of semistructured data based on information-theoretic approaches. The VLDB Journal, 21(5), 677–702.

\bibitem{2} Data Duplication

\bibitem{3} Ursin Brunner and Kurt Stockinger. 2020. Entity Matching with Transformer Architectures - A Step Forward in Data Integration. In EDBT. OpenProceedings.org, 463--473.
\url{https://doi.org/10.21256/zhaw-19637}

\end{thebibliography}
\end{document}
